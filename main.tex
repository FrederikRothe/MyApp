\documentclass{article}
\usepackage[utf8]{inputenc}
\usepackage{graphicx}

\title{Assignment 0}
\author{Frederik Rothe}
\date{September 2021}

\begin{document}

\maketitle

\section*{IsLeapYear()}

The IsLeapYear() method contains a very simple algorithm determining whether the year that the user inputs is a leap year. 
This is done with the use of three if-statements. \\
Firstly, we check whether the number is greater than 1582, if not return false.
Then, we check if the number is divisible by four, if not we discard it right away and return false. However, if the number is in fact divisible by four we have a candidate for a leap year. \\
Hereafter, we have the last if-statement to determine if that candidate really is a leap year. The rules of leap years say that years divisible by 100 is not a leap year, except \textbf{iff} the year is divisible by 400, then it is in fact a leap year. \\
This concludes our algorithm.
\begin{center}
\includegraphics[width=350, height=200]{IsLeapYearDiagram.jpg}    
\end{center}


\end{document}
